\section{Future Prospects}

This study opens several pathways for further investigation into the effects of data trimming on genomic analyses. Highlighting potential areas of research, we propose:

\paragraph{Impact on Structural Variations} Future work could explore how trimming influences the detection of genomic structural variations, crucial for understanding genomic dynamics.

\paragraph{Gene Annotation Quality} Investigating the effect of trimming on gene annotation and function prediction could refine genomic data interpretation.

\paragraph{Antibiotic Resistance Detection} Assessing how trimming adjustments affect identifying antibiotic resistance genes may offer insights into combating antimicrobial resistance.

\paragraph{Comparative Genomic Features} Analysis of trimming's role in identifying unique versus conserved genomic regions could enhance evolutionary and comparative genomic studies.

\paragraph{Mutation Detection Accuracy} Examining how trimming impacts the precision of mutation detection, especially in complex regions, could improve genetic analyses.

\paragraph{Assembly Efficiency} Research could evaluate how different trimming approaches optimize genome assembly, affecting speed and accuracy.

\paragraph{Metagenomic Differentiation} Studying trimming's effect on differentiating metagenomic samples could advance microbial genomics and environmental biology.

Each direction promises to deepen our genomic understanding, offering a foundation for innovative tools and methods that enhance the precision and reliability of genomic research.
