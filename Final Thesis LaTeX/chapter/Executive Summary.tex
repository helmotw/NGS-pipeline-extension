\chapter*{Executive Summary}


The thesis delves into the impact of various \gls{genome} \gls{trimming} Strategies on the \gls{assembly} quality of Mycobacterium tuberculosis, a pathogen of high public health importance \cite{Kemal2018}. In the era of Next-Generation
Sequencing (\gls{ngs}), achieving accurate \gls{assembly} is vital for understanding genetic diversity, pathogenicity, and drug resistance. The trimming process, a crucial preprocessing step in \gls{ngs}, involves removing low-quality bases and adapter sequences to improve data quality. This study evaluates the impact of various trimming techniques on \gls{assembly} metrics. It focuses on creating a series of genomic assemblies for Mycobacterium tuberculosis, relevant to the genomic studies of other organisms as well, by utilizing \gls{trimming}, \gls{genome} Assembling, and \gls{assembly} Evaluation processes within a comprehensive software pipeline. The objective is to fine-tune the preprocessing steps to boost the reliability of genomic assemblies. The research employs parallel processing, which is both effective and scalable \cite{Vishwasrao2017}, to achieve this goal. Through multiple iterations and creating dozens of \gls{genome}s, the research uncovers a significant finding: the effectiveness of trimming strategies varies. The primary challenge is to minimize undefined nucleotides for continuous \gls{assembly} while maintaining the quality metrics of it. This insight is crucial for selecting trimming methods that not only improve the \gls{assembly} process but also preserve the integrity of the genomic data. It was found that the quality of some trimming methods significantly surpassed others.

