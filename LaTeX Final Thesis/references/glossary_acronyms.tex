\newglossaryentry{dna}{name={DNA},description={is a double-helical molecule that carries the genetic instructions used in the growth, development, functioning, and reproduction of all known living organisms. The structure comprises two strands forming a double helix, held together by base pairs: adenine with thymine, and cytosine with guanine. This genetic blueprint is crucial for sequencing as it determines the specific order of nucleotides, the foundation of genetic diversity and heredity}}

\newglossaryentry{sequencing}{name={sequencing},description={is the process of determining the precise order of nucleotides within a DNA molecule. It encompasses any method or technology used to determine the order of the four bases—adenine, guanine, cytosine, and thymine—in a strand of DNA. The advent of rapid DNA sequencing methods has greatly enhanced biological and medical research and discovery}}

\newglossaryentry{ngs}{name={NGS},description={refers to a set of high-throughput methodologies that enable rapid sequencing of large segments of DNA base pairs. NGS has revolutionized genomic research by allowing for the sequencing of entire genomes swiftly and economically, surpassing the capabilities of traditional Sanger sequencing. It’s particularly relevant in this study for its role in enabling comprehensive analysis of Mycobacterium tuberculosis genomes}}

\newglossaryentry{paired-end}{name={paired-end sequencing},description={is a technique used in next-generation sequencing (NGS) that involves sequencing both ends of a DNA fragment to generate high-quality, alignable sequence data. This method allows for the sequencing of both the forward and reverse ends of the fragments, providing two reads per fragment. Paired-end sequencing is invaluable for detecting insertions, deletions, and rearrangements, and for improving the assembly of complex genomes. In the context of this research, paired-end sequencing can enhance the resolution and accuracy of Mycobacterium tuberculosis genome assemblies by providing more information on the spatial relationships between DNA fragments, thus facilitating the identification of genomic variations and structural changes}}

\newglossaryentry{fastq}{name={FASTQ-format},description={ is a widely adopted text-based format used for storing both biological sequences, typically nucleotide sequences, and their corresponding quality scores, with both the sequence letter and quality score being encoded using a single ASCII character for efficiency; this format plays a pivotal role in high-throughput sequencing datasets, including those generated by Next-Generation Sequencing (NGS) technologies, making it an indispensable format for handling and analyzing initial data in this study, thereby contributing significantly to the accuracy and reliability of genomic analysis}}

\newglossaryentry{assembly}{name={genome assembly},description={refers to the computational process of reconstructing the complete genomic sequence from fragmented DNA sequences obtained through sequencing. This complex task involves piecing together overlapping sequences to reconstruct the original genome}}

\newglossaryentry{contig}{name={contig},description={is a term derived from "contiguous," indicating a sequence of DNA that has been assembled from overlapping fragments. Contigs represent regions of a genome that have been successfully reconstructed without any gaps}}

\newglossaryentry{scaffold}{name={scaffold},description={represents in genome assembly a higher-level structure that connects several contigs, which are contiguous sequences of DNA, by bridging them with known gaps of unsequenced DNA, often indicated by the letter 'N', thereby offering a more comprehensive representation of the genome's organization and aiding in the approximation of how these sequences are arranged within a chromosome}}

\newglossaryentry{metrics}{name={assembly metrics},description={are quantitative measures used to assess the quality and completeness of a genome assembly. These metrics include but are not limited to, the total length of the assembly, the number of contigs, and the N50 statistic, which reflects the contig length such that 50\% of the entire assembly is contained in contigs of this length or longer}}

\newglossaryentry{trimming}{name={trimming},description={is a preprocessing step in genome analysis where low-quality or non-informative sections of DNA sequences are removed. Trimming improves the quality of data for assembly and analysis by eliminating errors and biases, enhancing the precision and dependability of subsequent computational assessments. This process is essential for minimizing the introduction of artefacts and improving the overall integrity of genomic interpretations}}

\newglossaryentry{depth}{name={coverage depth},description={is a measure of how many times a particular nucleotide in the genome is represented in the sequencing data. Higher coverage depth increases the reliability of the assembly and the identification of genetic variants}}

\newglossaryentry{deviation}{name={normalized deviation},description={is a statistical measure used to compare the variation of a particular metric from a reference or expected value, adjusted for the scale of the data. It is often used to assess the consistency and quality of genomic data across different samples or conditions}}

\newglossaryentry{heatmap}{name={heatmap},description={is a graphical representation of data where values in a matrix are depicted as colors. Heatmaps are useful in genomics for visualizing complex datasets, such as gene expression levels or sequence similarities, allowing for easy identification of patterns and outliers}}

\newglossaryentry{bar}{name={bar plots},description={are a type of graph used to represent and compare discrete data or categorical variables. In genomic studies, bar plots can visualize a variety of data, such as the distribution of different genomic features or the abundance of various taxa in metagenomic samples}}

\newglossaryentry{scatter}{name={scatter plot},description={is a graphical method used to display and assess the relationship between two quantitative variables. In the context of genomics, scatter plots can illustrate correlations between different genetic or phenotypic variables, aiding in the identification of potential genetic markers or traits}}

\newglossaryentry{total length}{name={total length},description={is the sum of the lengths of all sequences in a genome assembly. It gives an overall size of the assembly, reflecting the extent of the genomic material covered}}

\newglossaryentry{gc}{name={GC (\%)},description={represents the percentage of guanine (G) and cytosine (C) bases in the genome assembly. This metric is crucial as GC content varies significantly across different organisms and can influence sequencing and assembly processes}}

\newglossaryentry{largest contigs}{name={largest contig},description={are the longest contiguous sequences in the assembly. The size of the largest contigs can indicate the assembly’s continuity, with longer contigs often signifying a higher-quality assembly}}

\newglossaryentry{n50}{name={N50},description={is a statistical measure that provides the length of the shortest contig or scaffold in the set of largest contigs or scaffolds that together cover at least 50\% of the assembly. It is a key indicator of assembly quality, reflecting the contiguity and reliability of the genomic assembly. The N50 metric is especially valuable in gauging the effectiveness of assembly algorithms in producing long, uninterrupted stretches of the genome, which are crucial for accurate gene annotation and the identification of genomic structures. A higher N50 value indicates that a greater portion of the genome is contained within fewer, longer contigs or scaffolds, signifying a more complete and coherent assembly. This measure helps researchers compare assembly performance across different datasets or sequencing technologies, aiming for assemblies that best represent the biological reality of the genome under study}}

\newglossaryentry{n90}{name={N90},description={is similar to N50 but more stringent; it represents the length of the contigs or scaffolds in the set of the largest sequences that cumulatively account for 90\% of the total assembly length}}

\newglossaryentry{l50}{name={L50},description={is the number of contigs or scaffolds needed to cover 50\% of the genome assembly. It complements the N50 metric by providing a count of the sequences involved}}

\newglossaryentry{l90}{name={L90},description={complements the N90 metric and represents the number of contigs or scaffolds required to cover 90\% of the genome assembly. A lower L90 value indicates a more contiguous assembly}}

\newglossaryentry{n's per 100 kbp}{name={number of N’s per 100 kbp},description={is a critical assembly metric that quantifies the number of undetermined nucleotides ('N') normalized over 100 kilobase pairs of the assembly, providing valuable information about the frequency of gaps within scaffolds or contigs and serving as an indicator of the assembly's completeness and overall quality in genomic analysis. A lower value of N’s per 100 kbp indicates a higher quality genome assembly with fewer gaps, suggesting a more continuous and complete representation of the genome. This metric is crucial in de novo genome assembly, where the goal is to reconstruct the genome from short sequencing reads without a reference sequence. It helps researchers evaluate the contiguity and integrity of assembled sequences, guiding improvements in assembly methods and the selection of more accurate and comprehensive genomic datasets for further analysis}}

\newglossaryentry{read}{name={read},description={refers to a sequence of nucleotides (the basic units of DNA, represented by the letters A, T, C, and G) that has been obtained from sequencing a DNA molecule. These reads are the raw data produced by DNA sequencing technologies, like Next-Generation Sequencing (NGS) machines, which capture millions of fragments of DNA from a sample, providing a comprehensive snapshot of its genetic composition. The accuracy and depth of coverage of these reads are critical for reliable genomic analysis, determining the resolution at which genetic variations can be identified. Consequently, sophisticated bioinformatics tools are employed to assemble these reads into a coherent sequence, facilitating the identification of genetic markers, mutations, and structural variations within the genome}}

\newglossaryentry{genome}{name={genome},description={is the complete set of genetic material, including all the genes and non-coding sequences, contained within a DNA. It serves as the instruction manual for the growth, development, functioning, and reproduction of that organism. Genomes vary in size and complexity among different species and are responsible for the genetic traits and characteristics of individuals within those species}}



